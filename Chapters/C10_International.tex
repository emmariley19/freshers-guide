\chapterimage{chapter_head_2.jpg} % Chapter heading image

\chapter{Information for Overseas Students}
\enlargethispage{\baselineskip}

See also: \url{http://www.ox.ac.uk/students/new/international}

\section{Banking} 

The following banks have branches in central Oxford:
\begin{itemize}
  \item Barclays: 54 Cornmarket Street
  \item The Cooperative Bank: 13 New Road
  \item Halifax: 13A New Road
  \item HSBC: 65 Cornmarket Street  
  \item Lloyds: corner of High, Cornmarket, Queen, and St Aldate's Streets. 
  \item NatWest: 121 High Street and corner of Cornmarket and George Street
  \item Royal Bank of Scotland: corner of St Giles and Little Clarendon Street.
  \item Nationwide:44 Queen Street
  \item Santander: 114 St Aldate's Street
  \item TSB Bank: 17 George Street
\end{itemize}

The details of bank accounts change rapidly; an up-to-date comparison of student accounts can be found at: 
\url{http://www.savethestudent.org/money/student-banking/student-bank-accounts.html} 

Opening an account is surprisingly difficult. It is worth asking your bank at home if they can set up an account for you with an English bank (or another agency such as Thomas Cook in Australia). When opening your bank account, you will usually need:
\begin{itemize}
  \item A means of proving your identity and immigration status: your
passport with visa OR your EU national photo ID (whichever is applicable),
 \item Proof of your UK address: your admissions letter from College should do
 this as well (at least in the first weeks of term). Sometimes banks will be difficult about this, demanding bills which you logically will not yet have; this may be a sign that they are not the bank for you. 
 \item It may also be necessary to provide details of your previous bank
 accounts (i.e. a statement) 
\end{itemize}

Many of the banks will have stalls at the International Students' Orientation with exact descriptions of their current requirements.

Once you have opened an account, it can take weeks before your cheque book, debit card, cheque guarantee card (essential for payment by cheque), or credit card are available. Likewise, expect all deposits (except cash) to take up to one week before funds are made available to you. 

Bureau de Change counters are available in every bank for converting currencies.There are both American Express and Thomas Cook offices on Queen Street. Marks and Spencer on Queen Street also offers a currency exchange, as does the exchange service at the tourist information on Broad Street.

Many overseas students have found it difficult, costly, and time-consuming to
access funds from their home countries. Ideally, arrange to arrive in the UK
with a certified cheque issued by your bank already in Pounds Sterling for
however much of your money you wish to have available here. Most ATMs accept
overseas debit cards, allowing you to withdraw cash from your account back home,
but your bank will usually charge you for this. Credit cards are similarly
useful, although the fees and interest costs are potentially prohibitive. There
is also a policy of not issuing international students with a UK credit card
until they have been in the UK for at least 6 months. You can arrange with the
New College Bursary to pay your fees by international wire transfer. Information
on this process will be sent to you with your bill each time it is due. 

\section{Electricity and appliances} With the proper precautions and planning,
you should be able to bring most of your electrical appliances with you to Oxford. Be forewarned, however, that the UK's different electrical current and cycle rate has cost many the ill-prepared graduate some piece of much-loved equipment.

Electricity in the UK operates on a AC~220-240\,V, 50\,Hz system. If your
equipment is designed to run in a range that includes both these figures, you
simply need to purchase an adaptor which will allow you to fit your devices'
plug(s) to the wall outlets here. Check your devices' specifications. Many
recent computers, for instance, are designed to be used in 110-240\,V, 50-60\,Hz
ranges, thus requiring nothing more than an adaptor for use here.
These can be picked up at any number of stores in Oxford during your first weeks
here (Boswells on Broad Street is a good all-purpose department store).

If your equipment is not rated for the UK electrical system, you will need to
purchase a transformer which will alter the electrical current used in the UK to
the appropriate current for your equipment. American products, for example, are
usually built for AC~110\,V, 60\,Hz. While almost all transformers will easily
handle the step down from 220\,V to 110\,V, only very expensive ones will change
the cycle rate, ie the 50\,Hz to 60\,Hz. Any equipment that needs to run
constantly at a certain speed from an internal motor (e.g. CD players, blenders, cassette tape players, and clocks) may not work here, and may be damaged through trying them. Again, if in doubt, check your products' specifications and inquire at the dealer or manufacturer.

If your equipment will work with a transformer, make sure you get one that is powerful enough for everything that you plan to use. Transformers can only handle a certain amount of wattage. Add up the total amount of watts used by all of your equipment and make sure it is less than your transformer can handle, or you'll fry your transformer and potentially destroy your equipment. The Oxford University Computing Centre at 13 Banbury Road sells transformers although you would be best advised to pick them up in your home country before you arrive.

If you are living in College, there are a wide range of restrictions on your use
of electrical appliances: see the
\href{http://www.new.ox.ac.uk/sites/default/files/sites/all/files/HB_Web version_100915.pdf}{\urlformat{Dean's Handbook}}, 4f (pg. 27).

\section{Work} 
The \href{http://www.ox.ac.uk/about/international-oxford}{\urlformat{International Student Office}} which runs an Orientation programme for all international students at the start of your Oxford career, will be your best and primary resource for advice on visas, work permits, funding, etc.
Citizens of the UK, EEA and Switzerland have no work restrictions. Holders of
Tier 4 visas are restricted - your passport sticker should state the exact
restrictions (see here:
\url{http://www.ukcisa.org.uk/Information--Advice/Working/Can-you-work layer-5316}). These restrictions apply to paid teaching or pastoral work undertaken for a College or the University.
The UK government has a scheme for doctoral students to stay in the UK to work
after finishing a degree, and full details can be found on their website:
\url{http://www.ukcisa.org.uk/Information--Advice/Working/Working-after-studies layer-3780/}.
The \href{http://www.careers.ox.ac.uk/}{\urlformat{University Careers Service}} offers sessions and resources on working internationally or staying to work in the UK.

\section{Health care}
See also 
\begin{itemize}
  \item \url{http://www.new.ox.ac.uk/international-visiting-students}
  \item \url{http://www.nhs.uk/NHSENGLAND/Pages/NHSEngland.aspx}
\end{itemize}

New College has its own GP (medical doctor) on 28~Beaumont Street; you will be signed up for this service during College orientation.  
NHS provides free emergency care for all. However, unless you are a citizen of
the UK, the EEA, Switzerland, Australia, New Zealand or the Falkland Islands,
you will have to pay a \emph{Health surcharge} of \pounds150 per year, as part of
your visa in order to get access to non-emergency NHS services. You cannot opt out of this.
With that paid, you will have a right to free hospital care, free visits to your
GP, subsidised prescriptions (\pounds8.20 for most medicine), and subsidised
dental care. Sight tests, contact lenses and glasses are not covered by NHS.


\section{Mobile phones} 

The UK is a very mobile-phone-centred nation and you will really need a mobile to stay in touch with friends while here. Most Brits use texts (also known as SMS) or phone-based internet chat as their main form of communication, since these are cheaper than calls. 
If you don't bring a telephone to Oxford from your home country which will work in the UK, you can purchase a phone at any of the telephone retailers - most of which are located on Cornmarket Street (Vodafone, Orange, O2, etc). These retailers can also provide you with telephone service for your mobile. There are two main types of payment for mobile phone usage:

\begin{enumerate}
  \item \textbf{Pay as you go plan.} This means that you deposit money
towards your phone account and can make calls and texts until your money runs out.
 \item \textbf{Pay monthly.}  You pay a certain amount a month, for which you
get set amounts (sometimes unlimited) of minutes for calls, texts, and internet
allowance. The amount of each you get depends on how much you are willing to pay
per month! Without a bank account and/or proof of steady income, you will most
likely not qualify for the \emph{pay monthly} option, and will have to purchase a
\emph{pay as you go} plan. It is highly recommended therefore that, after you
have your bank account set up, you go and sign up for a monthly contract rather than a pay as you go plan.
\end{enumerate}


