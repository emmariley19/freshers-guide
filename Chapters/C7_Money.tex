\chapterimage{chapter_head_3i.jpg} % Chapter heading image

\chapter{Money}

Everything to do with money in College is in some way connected to the Bursary,
located on the ground floor of staircase~4OB. It is open weekdays between
9.30\,am and 12.30\,pm, and again between 2.15\,pm and 3.30\,pm. This is where you go to pick up grant cheques and other such payments. You can also go here to pay your battels at the beginning of each term and to add money to your till account (Bod card), but these processes are more easily done online via \url{food.new.ox.ac.uk}. You can also make payments through the golden letterbox in the wall, even when the Bursary is not open. If you have issues with payments, you can contact Linda Goodsell (\href{mailto:linda.goodsell@new.ox.ac.uk}{\urlformat{linda.goodsell@new.ox.ac.uk}}) or visit during opening hours. Bear in mind that the bursary is extremely busy in the first few weeks of term.

\section{Battels}
Have you been skipping over sentences with the word battels in them? If so, then
this paragraph is for you. Battels are your bill for accommodation, dinners and
other little things, such as some MCR events; you have to pay it at the start of each term. Accommodation is paid in advance at the beginning of the term, but dinners and other small expenses are not charged until the beginning of the next term. Your first battels will be put in your pidge, but after this it will be e-mailed to you. It is usually possible to negotiate a short extension to the payment deadline if it's really necessary. Any Junior Member who has an outstanding battels debt at Noon on the Friday of 1st Week,  and  who  has  not  seen  the  Bursar  or  emailed the  Bursary  to  agree  a timetable  for  settlement  of  this  debt,  will  be  required  to  pay  an  administrative charge of £5 and will be barred from further credit facilities within College.  Further payments will be imposed if Battels are still outstanding at Noon on Friday of 2nd Week. Battels can be paid in person at the bursary or online with your debit card (or credit card at a 1.92\% surcharge) through the \url{food.new.ox.ac.uk} website.
\section{Till account}
This is the account you use to pay for food in hall and drinks in the college
bar, which you do with your Bod-card. You \emph{CAN NOT} go in debt in the
college and MCR bars, unlike in the hall, where you can go in debt without
interests for breakfast and lunch up to \pounds15: this debt will be added to your next Battels. The easiest way top up
your account is online using the \url{food.new.ox.ac.uk} website and your debit card, however, you can also pay by cheque to the bursary.

\section[Grants and bursaries]{Travel grants, hardship grants and bursaries}
College has a research fund for graduate students. This is primarily for necessary travel (e.g. conference attendance/archival visits), but some purchases (e.g. essential software) will be considered on their merits. Taught master's students can get up to \pounds200 and research students can get \pounds375 per year (and this can accrue if not used). Medics on Electives can apply for special travel grants: \pounds750 per year for placement outside the U.K. Forms are on the college website at \url{www.new.ox.ac.uk/graduate-awards}.

There are numerous other funds for various things, particularly sport and other
\emph{meritorious} activities which will be advertised throughout the year in communication by college.. If, and only if, your circumstances change
adversely after you get to Oxford, then you are eligible for a hardship grant from The College Financial Aid Committee. Make an appointment with the bursar, who is usually very helpful if you are in genuine need.

Outside of College, those experiencing unexpected financial difficulties can also apply to the University Hardship Fund (\url{http://www.ox.ac.uk/students/fees-funding/assistance/hardship/uhf}) by the fourth week of each term. UK students can also apply to the Access to Learning Fund (\url{thttp://www.ox.ac.uk/students/fees-funding/assistance/hardship/alf}). The University website has further information on possible sources of financial support at \url{www.ox.ac.uk/feesandfunding/graduates} and OUSU offers an advice service (\url{advice@ousu.org}).

There are some other bursaries and small pots of money which may be available. a list of these is on the college website at \url{www.new.ox.ac.uk/graduate-awards}. You can apply for them via the Bursar, David Palfreyman (\href{mailto:bursar@new.ox.ac.uk}{\urlformat{bursar@new.ox.ac.uk}})

\section{Earning money}
You are primarily in Oxford to study and your supervisors will expect you to spend most of your time learning or doing research. However, there are opportunities to do part-time work. There are part-time opportunities in town, shifts in the college library, etc.

There are also teaching opportunities, although you are not required to teach,
and no one is guaranteed to get a teaching opportunity. Graduate students can
take undergraduate tutorials and some may even be appointed to \emph{college
lectureships}. Scientists also have the opportunity to demonstrate in practical sessions. Talk to your supervisor if you are interested in teaching.
